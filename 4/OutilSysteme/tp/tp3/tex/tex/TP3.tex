%\documentclass[11pt,answers]{exam}
\documentclass[11pt]{exam}

\usepackage{ntheorem}
\RequirePackage{amsmath,amssymb,amsfonts,natbib}
\usepackage[a4paper,margin={.1\paperheight,.1\paperwidth},marginratio=1:1]{geometry}
\usepackage{enumerate,epsfig}
\usepackage[T1]{fontenc}
%\usepackage[latin1]{inputenc}  %Pour Windows
\usepackage[utf8]{inputenc}   %Pour MAC 
\usepackage[french]{babel}
\usepackage{aeguill}
\usepackage{stmaryrd}
\usepackage{hyperref}
\usepackage{xcolor}
\usepackage[left]{lineno}
\newtheorem{definition}{D\'efinition}
\newtheorem{notation}{Notation}
{\theorembodyfont{\rmfamily}\newtheorem{example}{Exemple}}
{\theorembodyfont{\rmfamily\small}\newtheorem{exercise}{Exercice}}
\newtheorem{lemma}{Lemme}
\newtheorem{proposition}{Proposition}
\newtheorem{corollary}{Corollaire}
{\theorembodyfont{\rmfamily}\newtheorem{remark}{Remarque}}
\newtheorem{theorem}{Th\'eor\'eme}
{\theorembodyfont{\rmfamily}\newtheorem{warning}{Avertissement}}


% % % %
% Code Python
% % % %
\usepackage{listings}
%\usepackage{newalg}
\usepackage{color}

\definecolor{mygreen}{rgb}{0,0.6,0}
\definecolor{mygray}{rgb}{0.5,0.5,0.5}
\definecolor{mymauve}{rgb}{0.58,0,0.82}

\lstset{ %
  backgroundcolor=\color{lightgray},   % choose the background color; you must add \usepackage{color} or \usepackage{xcolor}
  basicstyle=\footnotesize,        % the size of the fonts that are used for the code
  breakatwhitespace=false,         % sets if automatic breaks should only happen at whitespace
  breaklines=true,                 % sets automatic line breaking
  captionpos=b,                    % sets the caption-position to bottom
  commentstyle=\color{mygreen},    % comment style, put "white" for hiding inline comments
  deletekeywords={...},            % if you want to delete keywords from the given language
  escapeinside={\%*}{*)},          % if you want to add LaTeX within your code
  extendedchars=true,              % lets you use non-ASCII characters; for 8-bits encodings only, does not work with UTF-8
 % frame=single,                    % adds a frame around the code
  keepspaces=true,                 % keeps spaces in text, useful for keeping indentation of code (possibly needs columns=flexible)
  keywordstyle=\color{blue},       % keyword style
  language=Python,                 % the language of the code
  morekeywords={*,...},            % if you want to add more keywords to the set
  numbers=left,                    % where to put the line-numbers; possible values are (none, left, right)
  numbersep=2pt,                   % how far the line-numbers are from the code
  numberstyle=\tiny\color{mygray}, % the style that is used for the line-numbers
  rulecolor=\color{black},         % if not set, the frame-color may be changed on line-breaks within not-black text (e.g. comments (green here))
  showspaces=false,                % show spaces everywhere adding particular underscores; it overrides 'showstringspaces'
  showstringspaces=false,          % underline spaces within strings only
  showtabs=false,                  % show tabs within strings adding particular underscores
  stepnumber=0,                    % the step between two line-numbers. If it's 1, each line will be numbered
  stringstyle=\color{mymauve},     % string literal style
  tabsize=2,                       % sets default tabsize to 2 spaces
  title={\color{mygreen}\it \lstname},                   % show the filename of files included with \lstinputlisting; also try caption instead of title
 %  morecomment=[is]{"""}{"""}  %  hide the comments 
}


% % % %
% % % %
%\rhead{\textsf{Licence 2 MASS-Math  \\2014-2015}} 
%\lhead{{\sf  AP3 Algorithmique et programmation \\ TD 
%\notd
%}}
\cfoot{\thepage}


% % % %
\newcommand{\N}{\mathbb{N}}
\newcommand{\Z}{\mathbb{Z}}

\usepackage{eurosym}

\usepackage{xspace}% espace intelligent pour les macros
\usepackage{../laTEX/macro}

\begin{document}

\noindent Université de Lorraine\hfill 
L1 M-I-SPI\bigskip

\begin{center}
{\large \bf Outils système}
\smallskip

{\large \bf TP 3}
\end{center}
\doublerulefill

Tout ce TP doit être fait avec vim.

\exercice{Les bases}
	\begin{enumerate}
		\item Faites un make générique pour java. Il s'utilisera de trois façon :
		\begin{itemize}
			\item {\tt make nomClass} et compilera avec javac la classe correspondante.
			\item {\tt make test nomClass}. Il compilera avec javac la classe correspondante et la testera (compilera et lancera le fichier {\tt test/TestnomClass.java}).
			\item {\tt make}. Il compilera et lancera le fichier principal {\tt Main.java}.
		\end{itemize}
		Vous pouvez tester votre {\tt makefile} avec un de vos projets java ou avec celui fourni sur Arche.
		
		\item Prenez connaissance du projet C disponible sur Arche. Deux modules (affichage et jouer), un fichier main et un header seul. 
		
		Les modules sont à compiler avec l'option {\tt -c} de {\tt gcc} (permettant de compiler des fichiers sans {\tt main}).
		
		Le main est à compiler avec les fichiers objets des modules ({\tt gcc main.c affichage.o jouer.o -o main}).
		
		{\tt doxygen} est une commande permettant de créer une documentation. Il s'utilise sur le fichier executable final.
		
		Votre make doit :
		\begin{itemize}
			\item Pouvoir compiler les modules (et tous les modules tant qu'à faire)
			\item Compiler le main
			\item créer la documentation
			\item Nettoyer le dossier en enlevant tous les fichiers objets (pas le fichier exécutable du main).
			\item Exporter l'exécutable et la bibliographie sous format zippé.
		\end{itemize}
		\item Faire un make pour les fichiers tex. Vous en trouverez un sur internet. Il génère plusieurs fichiers auxiliaires. Ajouter une commande à l'application de la règle utilisée pour générer {\tt fichierTex.tex} qui supprime tous les fichiers {\tt fichierTex.*} sauf le {\tt .tex} et le {\tt .pdf}.
	\end{enumerate}
	
\exercice{Make automatique}
	 Créez une commande {\tt touchMake} qui attend deux arguments : {\tt nomFichier}, {\tt langage}. Elle crée un nouveau fichier {\tt nomFichier}. Si l'argument {\tt langage} est présent, elle ajoute une règle au {\tt makefile}, en ayant préalablement listé les fichiers objets dont elle a besoin en cas de {\tt c}, (les {\tt \#include "..."}). S'il n'y a pas de {\tt makefile} elle doit le créer.
	 
	 Vous pouvez aussi rajouter le texte de base du langage (la classe en java, le main et {\tt \#include <stdio.h>,\#include <stdlib.h>} en c etc...)
	 
	 Les langages acceptés seront : {\tt c}, {\tt java}, {\tt python}, {\tt tex}.
	
	
\exercice{Make à la main}
	Faites une commande {\tt makeFST} mimant le make : elle lira un fichier {\tt makefile} donnée en argument (ou nommé {\tt makefile} par défault), le lira en repérant les règles. Pour chaque règle, elle cherchera les fichiers demandés (cibles et  fichiersRequis). Si les fichiers requis sont plus jeunes que la cible, elle exécutera le commande demandé. Sinon elle appliquera les règles à ces fichiers requis si besoin.
\end{document}
