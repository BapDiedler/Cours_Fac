\documentclass{article}
\usepackage[utf8]{inputenc}
\usepackage{lmodern} % Utilisation de la police Latin Modern
\usepackage[T1]{fontenc} % Utilisation de l'encodage T1
\usepackage{amsmath}
\usepackage{algorithm}
\usepackage{algorithmic}
\usepackage{hyperref}

\title{La Recherche Dichotomique}
\author{Diedler Baptiste}
\date{\today}

\begin{document}

\maketitle

\section{Introduction}
La recherche dichotomique, également connue sous le nom de recherche binaire, est un algorithme efficace pour rechercher un élément dans un tableau trié. Elle divise le tableau en deux moitiés à chaque étape, ce qui permet de réduire considérablement le nombre d'éléments à rechercher. 

Dans cet article, nous allons expliquer le fonctionnement de la recherche dichotomique et fournir un exemple de code en Python pour l'illustrer.

\section{Algorithme de Recherche Dichotomique}

L'algorithme de recherche dichotomique fonctionne comme suit :

\begin{algorithm}
\caption{Recherche Dichotomique}
\begin{algorithmic}
\REQUIRE Un tableau trié $arr$, un élément à rechercher $x$
\ENSURE L'index de l'élément $x$ dans le tableau, ou -1 si l'élément n'est pas présent
\STATE $gauche \gets 0$
\STATE $droite \gets \text{longueur}(arr) - 1$
\WHILE{$gauche \leq droite$}
    \STATE $milieu \gets \lfloor (gauche + droite) / 2 \rfloor$
    \IF{$arr[milieu] == x$}
        \RETURN $milieu$
    \ELSIF{$arr[milieu] < x$}
        \STATE $gauche \gets milieu + 1$
    \ELSE
        \STATE $droite \gets milieu - 1$
    \ENDIF
\ENDWHILE
\RETURN -1
\end{algorithmic}
\end{algorithm}

L'algorithme commence par initialiser deux indices, $gauche$ et $droite$, pour délimiter la plage de recherche. Ensuite, il calcule l'indice $milieu$ et compare la valeur $arr[milieu]$ avec l'élément recherché $x$. En fonction de la comparaison, l'algorithme ajuste les indices $gauche$ et $droite$ pour réduire la plage de recherche jusqu'à ce que l'élément soit trouvé ou que la plage soit vide.

\section{Exemple en Python}
Voici un exemple de code Python illustrant l'utilisation de la recherche dichotomique pour trouver un élément dans un tableau trié :

\begin{verbatim}
def recherche_dichotomique(arr, x):
    gauche, droite = 0, len(arr) - 1
    while gauche <= droite:
        milieu = (gauche + droite) // 2
        if arr[milieu] == x:
            return milieu
        elif arr[milieu] < x:
            gauche = milieu + 1
        else:
            droite = milieu - 1
    return -1

# Exemple d'utilisation
tableau = [1, 3, 5, 7, 9, 11, 13]
element = 7
resultat = recherche_dichotomique(tableau, element)
if resultat != -1:
    print(f"L'élément {element} se trouve à l'indice {resultat}.")
else:
    print(f"L'élément {element} n'a pas été trouvé.")
\end{verbatim}

\section{Conclusion}
La recherche dichotomique est un algorithme efficace pour la recherche d'éléments dans un tableau trié. Elle divise le tableau en deux à chaque itération, ce qui permet de réduire rapidement la plage de recherche. La complexité temporelle de cet algorithme est logarithmique par rapport à la taille du tableau.

Il est important de noter que la recherche dichotomique ne fonctionne que sur des tableaux triés. Si le tableau n'est pas trié, il faudra le trier d'abord, ce qui peut avoir une complexité temporelle supérieure.

\section{Références}
- Wikipédia, « Recherche dichotomique », \url{https://fr.wikipedia.org/wiki/Recherche_dichotomique}

\end{document}