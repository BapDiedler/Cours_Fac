\documentclass{article}

\usepackage[utf8]{inputenc}
\usepackage[T1]{fontenc}
\usepackage[french]{babel}
\usepackage{ntheorem}
\usepackage{amsmath}
\usepackage{amssymb}
\usepackage[ a4paper, hmargin={2cm, 2cm}, vmargin={3cm, 3cm}]{geometry}

\usepackage{tikz}

\theoremstyle{plain}
\theorembodyfont{\normalfont}
\theoremseparator{~--}
\newtheorem{exo}{Exercice}%[section]

\pagestyle{empty}

\begin{document}
\begin{center}
\large\sc Feuille d'exercice 3
\end{center}

% EX 1
\begin{exo} Récurrence sur les formules
\begin{enumerate}
    \item \begin{align*}
        simpl(x \wedge y \Rightarrow z) &= \neg simpl(x \wedge y) \vee z \\
                            &= \neg(\neg(\neg x \vee \neg y)) \vee z
        \end{align*}

    \item \begin{align*}
        ht(P) &= 0   & \text{si P atomique} \\
        ht(\neg(P)) &= 1 + ht(P) \\
        ht(P \circ Q) &= 1 + \max(ht(P), ht(Q)) & \circ \in \left\{\vee, \wedge, \Rightarrow\right\}  
    \end{align*}

    \item Montrons que pour toutes formules P, $simpl(P)$ ne contient pas le symbole $\Rightarrow$:\\
        $\varphi(n)$: Pour toute formule $(P)$, Si $ht(P)$ = n alors $simpl(P)$ ne contine pas $\Rightarrow$

    \begin{description}
        \item[$\bullet$ cas de bases:] $\varphi(0)$: Si $ht(P) = 0$ $simpl(P)$ ne 
        rajoute jamais le symbole '$\Rightarrow$'

        \item[$\bullet$ cas de récursif:] Supossons $\varphi(n)$ vrai.

            Prenons P de hauteur $n + 1$: 
            
            3 cas: $C = \left\{P \wedge Q, P \vee Q, P \Rightarrow Q \right\}$ 

            Si $P' \in C$ d'après $\varphi(n)$ $simpl(P)$ et $simpl(Q)$ ne contient pas
            de symbole '$\Rightarrow$' (car P et Q sont de hauteur n) or $simpl(P')$ ne rajout jamais le symbole '$\Rightarrow$'
            donc $P'$ ne contient pas le symbole '$\Rightarrow$'
    \end{description}

    \item Montrond par récurrence structurelle que $simpl(P)$ ne contient 
    pas de $\Rightarrow$:
    
    \begin{description}
        \item[$simple(\top)$]: $a \vee \neg a$
        \item[$simple(\bot)$]: $\neg (a \vee \neg a)$
        \item[$simpl(p)$]: p
    \end{description}

    Supposons la propritété vrai pour $P_1$ et $P_2$
    \begin{description}
        \item[$simpl(\neg P_1)$]: $\neg (simpl(P_1))$ vrai car $simpl(P_1)$ ne 
        contient pas de $\Rightarrow$.
        \item[$simpl(P_1 \wedge P_2)$]: $\neg (\neg simpl(P_1) \vee 
                \neg simpl(P_2))$ vrai par H.P
        \item[$simpl(P_1 \vee P_2)$]: $simpl(P_1) \vee simpl(P_2)$ vrai par H.P
        \item[$simpl(P_1 \Rightarrow P_2)$]: $\neg simpl(P_1) 
            \vee simpl(P_2)$ vrai par H.P
    \end{description}

    \item Voir question 3.
\end{enumerate}
\end{exo}

% EX 2
\begin{exo} Structure arborescente des formules, définition récursive. \\
    $a := \neg P \Rightarrow Q \vee \neg(P \vee R)$
\begin{enumerate}
    \item $(\neg P) \Rightarrow (Q \vee \neg(P \vee R))$
    \item forme arborescente: \\
    \begin{tikzpicture}[level distance=1cm]
        \node {$\Rightarrow$}
            child {
                node {$\neg$} 
                    child {node {$P$}}}
            child {
                node {$\vee$}
                    child {node {$Q$}}
                child {node {$\neg$} 
                    child{ node{$\vee$}
                        child{ node {$P$}}
                        child{ node {$R$}}
                        }
                }
        };
    \end{tikzpicture}
    
    \item A est vrai quand: P est vrai ou quand Q est vrai ou P et R Faux.
    \item $P \wedge \neg Q$
    \item \begin{enumerate}
        \item $neg(A)$ est vrai quand: P est faut et Q est faut et P ou R est vrai.
        \item \begin{align*}
            neg(\top) &= \bot \\
            neg(\bot) &= \top \\
            neg(p) &= \neg p & \text{p une variable propositionnelle} \\
            neg(\neg P) &= P \\
            neg(P \wedge Q) &= neg(P) \vee neg(Q)\\
            neg(P \vee Q) &= neg(P) \wedge neg(Q)\\
            neg(P \Rightarrow Q) &= P \wedge neg(Q)\\
        \end{align*}
        \item évident ($\neg (P \wedge Q) \Leftrightarrow \neg P \vee \neg Q$) pareil pour $\vee$ \\
            $\neg(P \Rightarrow Q) \Leftrightarrow (P \wedge \neg Q)$
    \end{enumerate}
\end{enumerate}
\end{exo}

% EX 3
\begin{exo}Sous formules
\begin{enumerate}
    \item $\neg(p \vee (q \wedge r)) \Rightarrow (p \wedge q) \\
         \left\{p \wedge q, q \wedge r,
            p \vee (q \wedge r),
            \neg(p \vee(q \wedge r)),\neg(p \vee(q \wedge r)) \Rightarrow (p \wedge q)
            , p, q, r\right\}$
    \item \begin{align*}
        sf(\top) &= \left\{\right\} \\
        sf(\bot) &= \left\{\right\} \\
        sf(p) &= \left\{p\right\} \\
        sf(\neg P) &= sf(P) \cup \left\{\neg P\right\}  \\
        sf(P \wedge Q) &= sf(P) \cup sf(Q) \cup \left\{P \wedge Q\right\}\\
        sf(P \vee Q) &= sf(P) \cup sf(Q) \cup \left\{P \vee Q\right\}\\
        sf(P \Rightarrow Q) &= sf(P) \cup sf(Q) \cup \left\{P \Rightarrow Q\right\}\\
    \end{align*}

    \item 
\end{enumerate}
\end{exo}

% EX 4
\begin{exo}

\end{exo}

% EX 5
\begin{exo}

\end{exo}

% EX 6
\begin{exo}Modèles de relation, examen session 2 2014/15

\end{exo}

% EX 7
\begin{exo}
\begin{enumerate}
\item $\exists x_1, \ldots , x_n, \forall x,  
        x = x_1 \vee \ldots \vee x = x_{n}$ 
    au plus $n - 1$ éléments

\item $B_n = \forall x_1, \ldots , x_n, \exists x,  
x \neq  x_1 \wedge \ldots \wedge x \neq  x_{n}$ 
au plus $n - 1$ éléments
\end{enumerate}
\end{exo}

% EX 8
\begin{exo} La théorie des entiers de Peano: \vspace*{0.2cm}

$I \models \begin{cases}
    \forall x \neg 0 = S(x) \\
    \forall x, y, S(x) = S(y) \Rightarrow x = y
\end{cases}$
\begin{enumerate}
\item $a_n = Val_I(S^n(0))$ \\
        $\mathfrak{D} = \left\{a_n| n \in \mathbb{N}\right\}$ \\
    $a_n = a_p,  n < p$ n plus petit possible.
    \begin{itemize}
        \item $n = 0$: $Val(0) = Val(S^p(0))$ \\ avec $p > 0$
                    $x \mapsto Val(S^{p-1}(0))$ impossible
        \item $n > 0$: \begin{align*}
            Val(S^n(0)) &= Val(S^p(0)) \\
            Val(S^{n-1}(0)) &= Val(S^{p-1}(0)) \\
            Val(S(S^{n-1}(0))) &= Val(S(S^{p-1}(0))) \\
            Val(S^{n-1}(0)) &= Val(S^{p-1}(0))\\
            a_{n-1} &= a_{p-1}
        \end{align*}
    \end{itemize}
\item $\mathfrak{D} = \left\{0\right\} S(x) = 0$
\item $\mathfrak{D} = \left\{0, 1\right\} S(x) = 1$ 
\item 
\end{enumerate}
\end{exo}

% EX 9
\begin{exo}

\end{exo}

% EX 10
\begin{exo} Logique monadique, examen 2017/18

\begin{enumerate}
\item $\tau (1) = (F, F), \tau (2) = (V, F), \tau (3) = (F, V)
    , \tau (4) = (V, F), \tau (5) = (F, F), \tau (6) = (V, V)$

\item Il peut y avoir 4 valeurs différentes dans l'interprétation N \\
    dans une interprétation quelquconque 
    il y a maximum 4 et minimum 1 valeurs différentes

\item $val(i, A) = val(i', A)$
    Supposon vrai $A := val(i, A) = val(i', A)$
    \begin{itemize}      
    \item \begin{align*}
        val(i, \neg A) &= neg(val(i, A)) \\
                        &= neg(val(i', A)) \\
                        &= val(i', A)
        \end{align*}
    \item \begin{align*}
        val(i, A \wedge B) &= et(val(i, A), val(i, B)) \\
                        &= et(val(i', A), val(i', B)) \\
                        &= val(i', A \wedge B)
        \end{align*} pareil pour $\vee$ mais avec la fonction $ou()$
    \item \begin{align*}
        val(i, P(X)) = & x \mapsto i(x) \vDash P(x) \\
                       &  x \mapsto i'(x) \vDash P(x)
        \end{align*}
    
    \item \begin{align*}
        I, i \vDash \forall x, A &\Leftrightarrow \forall d \in \mathcal{D}, I, i + \left\{x \mapsto d \right\} \vDash A \\
                                & \Leftrightarrow \forall d \in \mathcal{D}, I, i' + \left\{x \mapsto d \right\} \vDash A 
                                    & \text{par H.P et } i + \left\{x \mapsto d \right\} \backsimeq i' + \left\{x \mapsto d \right\} \\
                                &\Leftrightarrow I, i' \vDash \forall x, A
        \end{align*}
    \end{itemize}

\item \begin{enumerate}
    \item 
\end{enumerate}
\end{enumerate}
\end{exo}

\end{document}