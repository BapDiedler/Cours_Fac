\documentclass{article}

\usepackage[utf8]{inputenc}
\usepackage[T1]{fontenc}
\usepackage[french]{babel}
\usepackage{ntheorem}
\usepackage{amsmath}
\usepackage{amssymb}
\usepackage[ a4paper, hmargin={2cm, 2cm}, vmargin={3cm, 3cm}]{geometry}

\usepackage{tikz}
\usepackage{xcolor}

\theoremstyle{plain}
\theorembodyfont{\normalfont}
\theoremseparator{~--}
\newtheorem{exo}{Exercice}%[section]


\pagestyle{empty}

\begin{document}
\begin{center}
\large\sc Feuille d'exercice 2
\end{center}

\begin{exo} Vrai ou Faux \\
    \begin{tabular} {|c|}
        \hline F \\ \hline V \\ \hline
        \hline V \\ \hline F \\ \hline F \\ \hline
        \hline V \\ \hline F \\ \hline F \\ \hline
        \hline V \\ \hline F \\ \hline
        \hline F \\ \hline F \\ \hline F \\ \hline
    \end{tabular}
\end{exo}

\begin{exo}
\begin{enumerate}
    \item arbre:\\
    \begin{tikzpicture}
        \node at (-1, 0) {F:};
        \node {$\vee$} 
            [level distance=1cm,
            sibling distance=10mm,
            level 1/.style={sibling distance=20mm},
            ]

            child {
                node {$\exists x$} 
                    child {node {$p(x, f(x))$}}
                }
            child {
                node {$\neg$}
                child {node {$\forall y$} 
                    child{ node{$q(y, g(a, z, f(z)))$}}
                }
        };

        \node at (5, 0) {G:};
        \node at (6, 0) {$\vee$} [level distance=1cm,
            sibling distance=10mm,
            level 1/.style={sibling distance=25mm},
            level 2/.style={sibling distance=25mm},
            level 3/.style={sibling distance=10mm},
        ]
            child {
                node {$r(x)$} 
            }
            child {
                node {$\wedge$}
                    child {node {$\wedge$}
                        child { node {$\exists x$}
                            child { node {$\forall y$}
                                child {node {$p(f(x, z))$}}
                            }
                        }
                        child { 
                            node {$r(a)$}
                        }
                    }
                child {node {$\forall x$} 
                    child {
                        node{$q(y, g(x, z, x))$}
                    }
                }
        };
    \end{tikzpicture}

    \item  \textcolor{green}{Varaible libre}
    \begin{description}
        \item[$F$]: $(\exists \textcolor{cyan}{x}, p(\textcolor{cyan}{x}, 
                                                    f(\textcolor{green}{y}))) 
            \vee \neg \forall \textcolor{red}{y}, q(\textcolor{red}{y}, 
                                                        g(a, \textcolor{green}{z}
                                                        , f(\textcolor{green}{z})))$

        \item [$G$]: $r(\textcolor{green}{x}) \vee 
            ((\exists \textcolor{cyan}{x}, \forall \textcolor{yellow}{y}
                    , p(f(\textcolor{cyan}{x}), 
                            \textcolor{green}{z})) 
                \wedge r(a)) \wedge 
                    \forall \textcolor{red}{x}, q(\textcolor{green}{y}, 
                        g(\textcolor{red}{x}, \textcolor{green}{z}, \textcolor{red}{x}))$
    \end{description}

    \item Ce sont toutes les feuiilles des arbres de la question 1
    \item \begin{description}
        \item[$F$]: $x, f(x), y, g(a, z, f(z))$
        \item[$G$]: $x, f(x, z), y, g(x, z, x)$ 
    \end{description}
    \item  \begin{description}
        \item[$F$]: $(\exists x, p(x, f(f(a)))) \vee \neg \forall y, q(y, g(a, f(x), f(f(x))))$
        \item[$G$]: $r(x), \vee ((\exists w, \forall y, p(f(w), f(x))) \wedge r(a)) 
            \wedge \forall w, q(f(a), g(w, f(x), w))$ 
    \end{description}
\end{enumerate}
\end{exo}

\begin{exo} Arbre Binaire de Recherche 
    \begin{enumerate}
        \item   1: Oui $Node(Node(Node(Nil, 10, Nil), 15, Node(Nil, 20, Nil)), 33, Node(Nil, 40, Nil))$

                2: Non (le noeud 12 devrai être à droite du noeud 10)

        \item   $\max(n, t) := n \in t \wedge \forall x, x \in t \Rightarrow x = n \vee x < n$
        
        \item \begin{enumerate}
            \item $\forall x, \neg(x \in nil)$: Il n'y a pas de valeur dans une feuille
            \item $\forall$ $x$ $l$ $v$ $r, (x \in node(l, v, r) \Leftrightarrow  x \in l \vee x = v \vee x \ in)$:
                    Si x est dans un arbre alors il est soit dans la racine soit 
                    dans l'arbre de gauche soit dans l'arbre de droite
            \item $\forall t_1 t_2, \exists t, \forall x, (x \in t \Leftrightarrow x \in t_1 \wedge x \in t_2)$:
                    Il existe une intersection entre deux arbres.
        \end{enumerate}

        \item $abr(t)$ vrai si t est un Arbre Binaire de Rcherche

            \begin{enumerate}
                \item $abr(nil)$
                \item $\forall l, v, r, (abr(node(l, v, r)) \Leftrightarrow abr(l) \wedge abr(r) \\
                \wedge \forall x, (x \in l \Rightarrow x < v \wedge x \in r \Rightarrow v < x))$ 
            \end{enumerate}

        \item $\forall t,u, abr(t) \wedge abr(u) \Rightarrow (abr(union(t, u))
                \wedge \forall z, (z \in union(t, u) \Leftrightarrow z \in t \vee z \in u))$
    \end{enumerate}
\end{exo}

\begin{exo} Algorithmes satisfaibilité-validé

    $satisfiable(P) := non(valide(\neg P))$

    $valide(P) := non(satisfiable(\neg P))$
\end{exo}

\begin{exo} Interprétation en calcul des prédicats

    \begin{enumerate}
        \item formules: \\
        \begin{tabular}{l c c c c c}
                                                & 1 & 2 & 3 & 4 & 5 \\
            $\forall x y, P(x, y)$              & F & V & F & F & F \\
            $\exists x y, P(x, y)$              & F & V & V & V & V \\
            $\exists  x, \forall y, P(x, y)$    & F & V & V & F & F \\
            $\exists  y, \forall x, P(x, y)$    & F & V & F & V & F \\
            $\forall  x, \exists y, P(x, y)$    & F & V & F & V & V \\
            $\forall  y, \exists x, P(x, y)$    & F & V & V & F & F \\
        \end{tabular}

        \item 
        \begin{itemize}
            \item Toutes les cases sont noir
            \item le tableau à au moin une case noir
            \item Il y a une ligne noir
            \item Il y a une colonne noir
            \item il y a au moin une case noir par ligne
            \item Il y a au moin une case noir pas colonne
        \end{itemize}
    \end{enumerate}
\end{exo}

\begin{exo}
    Prouvez un résulatat théorique du théorème de la copacité
    \begin{enumerate}
        \item \begin{description}
            \item[Satisfiable:] Il exite une interpretation I, $A_i$ vrai dans I
            \item[Instasifiable:] Pour toutes interprétations I, au moins un
                                $A_i$ est faux dans $\mathcal{E}$   
        \end{description}

        \item Grâce à la compacité on peut dir que si $H \subset \mathcal{E} $ est instasifiable
                alors $\mathcal{E}$ est aussi instasifiable

        \item $\mathcal{E}$ est instasifiable donc pour toutes interpretation
                $I \in \mathbb{B}^{\mathbb{N}}$ il existe 
                $A_i \subset \mathcal{E}$ tel que $A_i$ faux dans I

                $H = \left\{A_i | i \in \mathbb{B}^{\mathbb{N}} \right\}$

        \item \textbf{Lemme de Köning:} Tout arbre infini à branchement 
            fini a une branche infinie.
    \end{enumerate}
\end{exo}

\begin{exo} Colirage de graphe
    \begin{itemize}
        \item Couleurs: $c \in C$
        \item $(x_i^c)$ vrai si $(x_i)$ est de couleur $c$
    \end{itemize}

    \[ Graphe
        \begin{cases}
            V = \left\{x_1, \ldots x_n\right\}  \\
            E \subseteq V^2 \\
            C = couleurs
        \end{cases}
        \]

    $n$ sommets et $k$ couleurs
    \begin{itemize}
        \item $n^k$
        \item 
        \begin{itemize}
            \item $\left\{x_i^c \Rightarrow \neg x_i^d | i \in V, c \neq d \in C\right\}$ 
            \item $\left\{x_i^c \Rightarrow \neg x_j^c | c \in C, (x_i, x_j) \in E\right\}$ 
            \item $\left\{x_i^{c_1}\vee \ldots \vee x_i^{c_n} | C = \left\{c_1, \ldots, c_n\right\}, x_i \in V\right\}$ 
        \end{itemize}
    \end{itemize}
\end{exo}

\end{document}