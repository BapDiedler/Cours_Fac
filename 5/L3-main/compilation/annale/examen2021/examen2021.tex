\documentclass{article}


\usepackage[utf8]{inputenc}
\usepackage[T1]{fontenc}
\usepackage[french]{babel}
\usepackage{listings}
\usepackage{xcolor}
\usepackage{ntheorem}
\usepackage{amsmath}
\usepackage{amssymb}
\usepackage{listings}
\usepackage{multirow}
\usepackage{subcaption}
\usepackage[ a4paper, hmargin={1cm, 1cm}, vmargin={2cm, 2cm}]{geometry}


\usepackage{tikz}
\usetikzlibrary{automata, arrows.meta, positioning}

\theoremstyle{plain}
\theorembodyfont{\normalfont}
\theoremseparator{~--}
\newtheorem{exo}{Exercice}%[section]

\lstset{language=caml, texcl=true, mathescape=true}

\title{Correction Examen 2021}
\author{Valeran MAYTIE}
\date{}

\begin{document}
\maketitle

\begin{exo} Rétro-ingénierie
\begin{enumerate}
  \item \begin{description}
      \item[Ocaml]$(e_1 = e_2) = e_3$

      \item[Python]: $(e_1 == e_2 == e_3)$ vérifie qu'ils soient tous égaux
  \end{description}

  \item \begin{description}
    \item[Ocaml]: le type de $e_1$ et  $e_2$ doivent être similair 
      et $e_3$ doit être de type $bool$

    \item[Python]: à l'aire d'accèpter n'importe 
      quelle combinaison de types
  \end{description}
\end{enumerate}
\end{exo}

\begin{exo}
\begin{enumerate}
  \item Grammaire G
    \begin{lstlisting}
        S ::= E#
        E ::= (L)
           |  sym
        L ::= $\varepsilon$
           |  E L
    \end{lstlisting}

  \item Déterminisation:\vspace{3mm} \\
    \begin{tikzpicture}
      \node(q0) [state, initial, rectangle] at (0, 0)
        {$S \to \bullet E \#$};
      \node(0) [draw, circle, fill=white, font=\footnotesize] 
              at (q0.north west) {0};
      \node(qf) [state, accepting, rectangle] at (3, 0)
        {$S \to E \bullet \#$};
      \node(1) [draw, circle, fill=white, font=\footnotesize] 
              at (qf.north west) {1};
      \node(q1) [state, rectangle, text width = 2cm] at (0, -2)
        {$E \to (\bullet L)$
        $L \to \varepsilon \bullet$};
      \node(2) [draw, circle, fill=white, font=\footnotesize] 
              at (q1.south west) {2};
      \node(q2) [state, rectangle, text width = 2cm] at (3, -4)
        {$L \to E \bullet L$
        $L \to \varepsilon \bullet$};
      \node(3) [draw, circle, fill=white, font=\footnotesize] 
              at (q2.north west) {3};
      \node(q3) [state, rectangle] at (6, -4)
        {$L \to E L \bullet$};
      \node(4) [draw, circle, fill=white, font=\footnotesize] 
              at (q3.north west) {4};
      \node(q4) [state, rectangle] at (3, -2)
        {$E \to sym \bullet$};
      \node(5) [draw, circle, fill=white, font=\footnotesize] 
              at (q4.north east) {5};
      \node(q5) [state, rectangle] at (0, -4)
        {$E \to (L\bullet)$};
      \node(6) [draw, circle, fill=white, font=\footnotesize] 
              at (q5.north west) {6};
      \node(q6) [state, rectangle] at (0, -6)
        {$E \to (L)\bullet$};
    \node(7) [draw, circle, fill=white, font=\footnotesize] 
              at (q6.north west) {7};

      \draw[->] (q0) -- (qf) node[above, pos=.5]{$E$};
      \draw[->] (q0) -- (q1) node[left, pos=.5]{$($};
      \draw[->] (q1.140) to [distance=10mm] 
        node[left, pos=.8]{$($} (q1.150);
      \draw[->] (q2.south) to [out=300,in=230,distance=10mm]
        node[left, pos=.8] {$E$} (q2.260);
      \draw[->] (q1.south) -- (q2.west) node[above, pos=.5]{$E$};
      \draw[->] (q2.north) -- (q1.east) node[below, pos=.5]{$($};
      \draw[->] (q2) -- (q4) node[right, pos=.5]{$sym$};
      \draw[->] (q2) -- (q3) node[below, pos=.5]{$L$};
      \draw[->] (q1) -- (q4) node[above, pos=.5]{$sym$};
      \draw[->] (q0) -- (q4) node[above, pos=.5]{$sym$};
      \draw[->] (q1) -- (q5) node[left, pos=.5]{$L$};
      \draw[->] (q5) -- (q6) node[left, pos=.5]{$)$};
    \end{tikzpicture}

  \item Il peut y avoir des conflits (reduction/transition) pour les blocs
    avec la règle $L \to \varepsilon \bullet$

    Si le prochain symbole est un symbole ou une parenthèse

    Exemple: $(L)$

  \item Table:\\
    \begin{tabular}[c]{l|| c|c|c|c || c | c | c ||}
      & \multicolumn{4}{ c ||}{Action} & \multicolumn{3}{c ||}{Sauts} \\
      & \# & ( & ) & sym & S & E & L \\
      \hline
    0 &    & 2 &   & 5   &   & 1 &   \\
      \hline
    1 & OK &   &   &     &   &   &   \\
      \hline
    \multirow{2}{*}{2} & & 2 & & 5 & & 3 & 6 \\
      & \multicolumn{4}{ c ||}{$L \prec \varepsilon$} & & & \\
      \hline
    \multirow{2}{*}{3} &   & 2 &   & 5    &   & 3 & 4 \\
      & \multicolumn{4}{c ||}{$L \prec \varepsilon$} & & & \\
      \hline
    4 & \multicolumn{4}{c ||}{$L \prec EL$} & & & \\
      \hline
    5 & \multicolumn{4}{c ||}{$E \prec sym$} & & & \\
      \hline
    6 &   &   & 7 & & & & \\
      \hline
    7 & \multicolumn{4}{c ||}{$E \prec (L)$} & & & \\
      \hline
    \end{tabular}

  \begin{table}
    \begin{subtable}[c]{0.5\textwidth}
      \centering
        \begin{tabular}{l|c|c|c}
            & S & E & L \\
            \hline
          0. & $\emptyset$ & $\emptyset$ & $\emptyset$ \\
          1. & $\emptyset$ & (, sym & $\emptyset$ \\
          2. & (, sym & (, sym & (, sym \\
          3. & (, sym & (, sym & (, sym \\
        \end{tabular}
      \subcaption*{Permier}
    \end{subtable}
    \begin{subtable}[c]{0.5\textwidth}
      \centering
        \begin{tabular}{l|c|c|c}
            & S & E & L \\
            \hline
          0. & $\emptyset$ & $\emptyset$ & $\emptyset$ \\
          1. & $\emptyset$ & \# & ) \\
          2. & $\emptyset$ & \#, (, sym, ) & ) \\
          3. & $\emptyset$ & \#, (, sym, ) & ) \\
        \end{tabular}
      \subcaption*{Suivant}
    \end{subtable}
  \end{table}
    
\end{enumerate}
\end{exo}

\newpage
\begin{exo} Compilation vers de valeurs optionnelles:
\begin{enumerate}
  \item Description:\\
    \begin{table}[htb]
    \centering
      \hfill
       \begin{tabular}{c | c}
        registre & valeurs \\
        \hline
        $\$a0$ & $8$\\
        $\$v0$ & $\texttt{0x10040008}$\\
        $\$t0$ & $1$
      \end{tabular}
      \hfill
      \begin{tabular}{l|c|}
        \cline{2-2}
        $\texttt{0x10040000}$ & 1\\
        & 2\\
        \cline{2-2}
        $\texttt{0x10040008}$ & 1\\
        & $\texttt{0x10040000}$\\
        \cline{2-2}
      \end{tabular}
      \hfill
    \end{table}

  \item $\$a0 = \texttt{0x10040000}$\\
    Tas:\\
    \begin{tabular}{l| c |}
      \cline{2-2}
      $\texttt{0x10040000}$ & 1 \\
      & $\texttt{0x10040008}$ \\ 
      \cline{2-2}
      $\texttt{0x10040008}$ & 1 \\
      & $\texttt{0x10040010}$ \\
      \cline{2-2}
      $\texttt{0x10040010}$ & 1\\
      & $3$\\
      \cline{2-2}
    \end{tabular}

  \item récupère la valeur $n$ de $\texttt{Somme (Somme n)}$  dans le registre $\$v0$
    \begin{lstlisting}[texcl=false, mathescape=false]
    lw $v0, 4($a0)
    lw $v0, 4($v0)
  \end{lstlisting}

  \item code de la fonction f:
    \begin{lstlisting}[texcl=false, mathescape=false]
      f:
        beqz  $a0, N
        lw    $v0, 4($a0)
        addi  $v0, $v0, 1
        jr    $ra
        N:
          li $v0, -1
          jr $ra
    \end{lstlisting}


  \item Code:
    \begin{lstlisting}[texcl=fasle, mathescape=false]
      bnez    $a0, some
      
      bnez    $a1, incompatible
      li      $v0, 1
      b       fin
    some:
      beqz    $a1, incompatible

      lw      $a0, 4($a0)
      lw      $a1, 4($a0)
      seq     $v0, $a0, $a1
      b fin
    incompatible:
      li      $v0, 0
    fin:
    \end{lstlisting}
\end{enumerate}
\end{exo}

\newpage
\begin{exo}Typage et sémantique\\
Partie sur le sémantique pas vue en cours.
\end{exo}

\end{document}
