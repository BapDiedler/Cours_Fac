\documentclass{article}

\usepackage[utf8]{inputenc}
\usepackage[T1]{fontenc}
\usepackage[french]{babel}
\usepackage{ntheorem}
\usepackage{amsmath}
\usepackage{amssymb}
\usepackage[ a4paper, hmargin={2cm, 2cm}, vmargin={3cm, 3cm}]{geometry}

\usepackage{tikz}
\usetikzlibrary{automata, arrows.meta, positioning}


\theoremstyle{plain}
\theorembodyfont{\normalfont}
\theoremseparator{~--}
\newtheorem{exo}{Exercice}%[section]


\pagestyle{empty}

\begin{document}
\begin{center}
\large\sc Feuille d'exercice 2
\end{center}

\begin{exo} $\left\{a, b\right\}$ notre alphabet
\begin{enumerate}
    \item $a{(a|b)}^*$: mots commençant par a
    \item $a^*|b^*$: mots n'ayant que des a ou que des b
    \item ${(b|ab)}^*(a|\varepsilon)$ mots n'ayant pas 2 a consécutifs
    \item ${(aa|b)}^*$: mots ayant des blocs de a de longueur paire
    \item ${(ab^*a|b)}^*$: mots ayant un nombre pair de a
\end{enumerate}
\end{exo}

\begin{exo} Expression régulières
\begin{enumerate} 
    \item $(\varepsilon|a|b)(\varepsilon|a|b)$
    \item $((a|b)(a|b))*$
    \item $a^*abb^*$
    \item $(a^*b^*)|(b^*a^*)$
    \item $b^*ab{(ab|b)}^*$
\end{enumerate}
\end{exo}

\begin{exo} Automates
\begin{enumerate}
    \item $/* \ldots */$ \\
    \begin{tikzpicture}
        %\draw [help lines] (-5,0) grid (5,1);

        \node (q0) [state, initial] at (-4.5, 0.5){};
        \node (q1) [state] at (-2.5, 0.5){};
        \node (q2) [state] at (-0.5, 0.5){};
        \node (q3) [state] at (1.5, 0.5){};
        \node (q4) [state, accepting] at (3.5, 0.5){};

        \path [-stealth, thick]
            (q0) edge node [above] {/}   (q1)
            (q1) edge node [above] {*}   (q2)
            (q2) edge node [above] {*}   (q3)
            (q3) edge node [above] {/}   (q4)
            (q2) edge [loop above] node {A$\backslash$*} ()
            (q3) edge [bend left] node [below] {A $\backslash$ '/'} (q2)
        ;
    \end{tikzpicture}

    \item nombres:\\
    \begin{tikzpicture}
        \node (q-) [state, initial] at (-4.5, 0){};
        \node (q0) [state,] at (-2.5, 0){};
            \node (q1) [state, accepting] at (-0.5, 2){};
            \node (q2) [state] at (1.5, 2){};
            \node (q3) [state, accepting] at (3.5, 2){};
        \node (q4) [state, accepting] at (-0.5, 0){};
            \node (q5) [state] at (-0.5, -2){};
            \node (q6) [state] at (1.5, -2){};
            \node (q7) [state] at (3.5, -2){};
            \node (q8) [state] at (5.5, -2){};
                \node (q9) [state] at (4.5, -4){};
                \node (q10) [state, accepting] at (6.5, -4){};

        \path [-stealth, thick]
            (q-) edge node [above] {$-$}   (q0)
            (q0) edge node [left] {1-9}   (q1)
            (q1) edge [loop above] node {0-9}   ()
            (q1) edge node [above] {.} (q2)
            (q2) edge [loop above] node [above] {0-9}   ()
            (q2) edge [] node [above] {1-9}   (q3)
            (q0) edge [] node [above] {0}   (q4)
            (q4) edge [] node [left] {.}   (q2)
            (q0) edge [] node [left] {1-9}   (q5)
            (q5) edge [] node [above] {.}   (q6)
            (q6) edge [loop above] node [above] {0-9}   (q7)
            (q6) edge [] node [above] {1-9}   (q7)
            (q7) edge [] node [above] {e}   (q8)
            (q8) edge [] node [left] {$-$}   (q9)
            (q8) edge [] node [right] {1-9}   (q10)
            (q9) edge [] node [above] {1-9}   (q10)
            (q10) edge [loop right] node [right] {0-9}   ()
        ;
    \end{tikzpicture}

    \item Toto
    \begin{tikzpicture}
        \node (q0) [state, accepting] at (-2.5, 0){$\cdot $};
        \node (q1) [state, accepting] at (-0.5, 0){$\cdot $};

        \path [-stealth, thick]
            (q0) edge [bend right] node [above] {$+$}   (q1)
        ;
    \end{tikzpicture}
\end{enumerate}    
\end{exo}

\begin{exo}Lemme de l'étoile:
\begin{enumerate}
    \item Par l'absurde on applique le lemme de l'étoile au 3 cas:
        \begin{itemize}
            \item $w_2=a^k$
            \item $w_2=a^{k_1}b^{k_2}$
            \item $w_3=b^k$
        \end{itemize}
        On arrive à une absurditée pour les 3.
    
    \item On va appliquer le lemme de l'étoile fort:\\
        Suppposons que L soit reconnaissable. On prend $x=a^{N-1} b a^{N-1}$
        On cherche tous les m de longueur $l \geq N$\\
        On à 3 cas:
        \begin{itemize}
            \item $m=a^{N-1} b$
            \item $m=ba^{N-1}$
            \item $m=a^{n_1}ba^{n_2}$ avec $n_1 + n_2 \geq N - 1$
        \end{itemize}
        Dans tous les cas on arrive à une absurditée.

    \item On va appliquer le lemme sur la fin de $baba^2\ldots ba^N$ similaire \ldots
    \item On prend $n$ premier plus grand ou égale que $N$. 
    On prend $a^n = a^{n_1} + a^{n_2} + a^{n_3}$ avec $n_2 \neq 0$ et $n_1 + n_2 + n_3 = n$
    d'après le lemme de l'étoile on a $\forall k, a^{n_1}a^{kn_2}a^{n_3} \in L$.

    Or $a^{n_1}a^{kn_2}a^{n_3}=a^{n_1}a^{n_3}a^{kn_2}=a^{n - n_2}a^{kn_2}$.
    On prend $k = n + 1$ on obtient $a^{(n + 1)n_2} \in L$.

    Or $(n + 1)(n_2)$ n'est pas premier.
\end{enumerate}
\end{exo}

\begin{exo} Réduction
\begin{enumerate}
\item On sait que le complémentaire d'un langage reconnaissable est aussi reconnaissable.

    Donc $L_1 \cap L_2 = \overline{\overline{L_1} \cup \overline{l_2}}$. 
    Or l'uniion de deux langage est reconnaissable $(e_1|e_2)$

\item On prend $L'=a^*b^*$. On Suposse que L est reconnaissable donc on à $L \cap L'$ reconnaissable
    Or $\left\{a^n b^n | n \geq 0\right\}$ n'est pas reconnaissable donc L ne l'ai pas.

\item On a $(Q, T, I, F)$ un automate qui reconnai $L_B$ 

    on peut donc construire 
    $(Q, T', I, F)$ avec $T' = \left\{(p, a, q)| p \overset{f(a)}{\longrightarrow} q \right\} $
\end{enumerate}
\end{exo}

\end{document}