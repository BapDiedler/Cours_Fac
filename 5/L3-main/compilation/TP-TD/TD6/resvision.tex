\documentclass{article}

\usepackage[utf8]{inputenc}
\usepackage[T1]{fontenc}
\usepackage[french]{babel}
\usepackage{ntheorem}
\usepackage{amsmath}
\usepackage{amssymb}
\usepackage[ a4paper, hmargin={1cm, 0.5cm}, vmargin={3cm, 3cm}]{geometry}
\usepackage{proof}

\usepackage{listings}
\usepackage{tikz}
\usetikzlibrary{automata, arrows.meta, positioning}


\theoremstyle{plain}
\theorembodyfont{\normalfont}
\theoremseparator{~--}
\newtheorem{exo}{Exercice}%[section]


\pagestyle{empty}

\begin{document}
\begin{center}
\large\sc TD Révision
\end{center}

\section{Analyse syntaxique}
\begin{enumerate}
  \item \begin{itemize}
    \item Some 1 + 2\\
      \begin{tikzpicture}
        \node(Some) at (0, 0) {Some};
        \node(1) at (1, 0) {1};
        \node(P) at (1.5, 0) {+};
        \node(2) at (2, 0) {2};

        \node(expr1) at (1, 0.6) {expr};
        \node(expr2) at (2, 0.6) {expr};
        \node(expr3) at (1.5, 1.5) {expr};
        \node(expr4) at (0.75, 2.5) {expr};

        \draw (1) -- (expr1) -- (1, 1) -- (2, 1) -- (expr2) -- (2);
        \draw (P) -- (expr3);
        \draw (Some) -- (0, 2) -- (1.5, 2) -- (expr3);
        \draw (0.75, 2) -- (expr4);

        \node(Some) at (10, 0) {Some};
        \node(1) at (11, 0) {1};
        \node(P) at (11.5, 0) {+};
        \node(2) at (12, 0) {2};

        \node(expr1) at (12, 0.6) {expr};
        \node(expr2) at (11, 0.6) {expr};
        \node(expr3) at (10.5, 1.5) {expr};
        \node(expr4) at (11.25, 2.5) {expr};

        \draw (Some) -- (10, 1) -- (11, 1) -- (expr2) -- (1);
        \draw (10.5, 1) -- (expr3);
        \draw (expr3) -- (10.5, 2) -- (12, 2) -- (expr1) -- (2);
        \draw (11.25, 2) -- (expr4);
      \end{tikzpicture}

    \item ? 1: 2 + 3\\
      \begin{tikzpicture}
        \node (QM) at (0, 0) {?};
        \node (1) at (0.5, 0) {1};
        \node (COL) at (1, 0) {:};
        \node (2) at (1.5, 0) {2};
        \node (P) at (2, 0) {+};
        \node (3) at (2.5, 0) {3};

        \node (expr1) at (0.5, 0.6) {expr};
        \node (expr2) at (1.5, 0.6) {expr};
        \node (expr3) at (2.5, 0.6) {expr};
        \node (expr4) at (2, 1.5) {expr};
        \node (expr5) at (1, 2.5) {expr};

        \draw (2) -- (expr2) -- (1.5, 1) -- (2.5, 1) -- (expr3) -- (3);
        \draw (P) -- (expr4);
        \draw (QM) -- (0, 2) -- (2, 2) -- (expr4);
        \draw (COL) -- (expr5);
        \draw (1) -- (expr1) -- (0.5, 2);

        \node (QM) at (10, 0) {?};
        \node (1) at (10.5, 0) {1};
        \node (COL) at (11, 0) {:};
        \node (2) at (11.5, 0) {2};
        \node (P) at (12, 0) {+};
        \node (3) at (12.5, 0) {3};

        \node (expr1) at (10.5, 0.6) {expr};
        \node (expr2) at (11.5, 0.6) {expr};
        \node (expr3) at (12.5, 0.6) {expr};
        \node (expr4) at (10.75, 1.5) {expr};
        \node (expr5) at (11.75, 2.5) {expr};

        \draw (QM) -- (10, 1) -- (11.5, 1) -- (expr2) -- (2);
        \draw (1) -- (expr1) -- (10.5, 1) -- (11, 1) -- (COL);
        \draw (10.75, 1) -- (expr4);
        \draw (expr4) -- (10.75, 2) -- (12.5, 2) -- (expr3) -- (3);
        \draw (P) -- (12, 2);
        \draw (11.75, 2) -- (expr5);
      \end{tikzpicture}

    \item 1 + 2 + 3

      \begin{tikzpicture}
        \node (1) at (0, 0) {1};
        \node (P1) at (0.5, 0) {+};
        \node (2) at (1, 0) {2};
        \node (P2) at (1.5, 0) {+};
        \node (3) at (2, 0) {3};

        \node (expr1) at (0, 0.6) {expr};
        \node (expr2) at (1, 0.6) {expr};
        \node (expr3) at (2, 0.6) {expr};
        \node (expr4) at (0.5, 1.5) {expr};
        \node (expr5) at (1.25, 2.5) {expr};

        \draw (1) -- (expr1) -- (0, 1) -- (1, 1) -- (expr2) -- (2);
        \draw (P1) -- (expr4);
        \draw (expr4) -- (0.5, 2) -- (2, 2) -- (expr3) -- (3);
        \draw (P2) -- (1.5, 2);
        \draw (1.25, 2) -- (expr5);

        \node (1) at (10, 0) {1};
        \node (P1) at (10.5, 0) {+};
        \node (2) at (11, 0) {2};
        \node (P2) at (11.5, 0) {+};
        \node (3) at (12, 0) {3};

        \node (expr1) at (10, 0.6) {expr};
        \node (expr2) at (11, 0.6) {expr};
        \node (expr3) at (12, 0.6) {expr};
        \node (expr4) at (11.5, 1.5) {expr};
        \node (expr5) at (10.75, 2.5) {expr};

        \draw (2) -- (expr2) -- (11, 1) -- (12, 1) -- (expr3) -- (3);
        \draw (P2) -- (expr4);
        \draw (1) -- (expr1) -- (10, 2) -- (11.5, 2) -- (expr4);
        \draw (P1) -- (10.5, 2);
        \draw (10.75, 2) -- (expr5);
      \end{tikzpicture}
  \end{itemize}

  \item Le plus prioritaire est l'option puis l'adition

    $ \texttt{\%nonassoc SOME}\\
        \texttt{\%left PLUS}\\ 
        \texttt{\%nonassoc COLON}$
    
\end{enumerate}

\section{Types}
\begin{enumerate}
  \item expression $? e_1 : e_2$ 
    \[
    \infer{\Gamma \vdash ?e_1 : e_2 : \tau}
    {\Gamma \vdash e_1 : \tau\text{ option} & \Gamma \vdash e_2 : \tau}
  \]

\item $\Gamma(x) = \text{int}, \Gamma(y) = \text{int option}$ 
  \begin{enumerate}
    \item Pas bien typée (int + int option) pas possible

    \item Bien typée (int)
    \[
      \infer{\Gamma \vdash x + ?y:42 : \text{int}}
      {
        \infer{\Gamma \vdash x : int}{\Gamma(x) = \text{int}}
      &
        \infer{\Gamma \vdash ?y:42 : \text{int}}
        {
          \infer{\Gamma \vdash y : \text{int option}}{\Gamma (y) = \text{int option}}
          &
          \infer{42 : int}{}
        }
      }
    \]

    \item Bien typée (int option)
    \[
      \infer{\Gamma \vdash ?\text{None}: (?\text{Some } y: \text{Some } x) : \text{int option}}
      {
        \infer{\Gamma \vdash \text{None} : \text{(int option) option}}{}
        &
        \infer{\Gamma \vdash ?\text{Some } y: \text{Some } x : \text{int option}}
        {
          \infer{\Gamma \vdash ?\text{Some } y: \text{(int option) option}}
            {\infer{y : \text{int option}}{\Gamma(y) = \text{int option}}}
          &
          \infer{\Gamma \vdash \text{Some } x : \text{int option}}
            {\infer{\Gamma \vdash x : int}{\Gamma(x) = int}}
        }
      }
    \]
  \end{enumerate}

\item Montrons ($\texttt{?Some e: Some d} \text{ bien typée} \Leftrightarrow 
  \texttt{Some (?a:d)} \text{ bien typée}$)

  \begin{description}
    \item[$\Rightarrow$]: Supossons $\Gamma \vdash \texttt{?Some e : Some d} : \tau$  
      est dérivable.

      Inversion: nécessairement, $\Gamma \vdash \texttt{Somme e} : \tau \text{ option}$

      et $\Gamma \vdash \texttt{Some d} : \tau$

      et $\Gamma \vdash \texttt{e} : \tau \text{ option}$

      et $\Gamma \vdash \texttt{d} : \alpha \text{ option} $ ($\tau$ = $\alpha$ option)

      \[
        \infer{\Gamma \vdash \texttt{Some(?e:d)} : \tau (= \alpha \text{ option})}
          {\infer{\Gamma \texttt{?e:d} : \alpha}{
            \infer{\Gamma \vdash \texttt{e} : \alpha \text{ option} = \tau}{\vdots}
            &
            \infer{\Gamma \vdash \texttt{d} : \alpha}{\vdots}
          }}
      \]

    \item[$\Leftarrow$]: Similaire
  \end{description}
\end{enumerate}

\section{Compilation}
\begin{exo} Compilation vers de valeurs optionnelles:
\begin{enumerate}
  \item Description:\\
    \begin{table}[htb]
    \centering
      \hfill
       \begin{tabular}{c | c}
        registre & valeurs \\
        \hline
        $\$a0$ & $8$\\
        $\$v0$ & $\texttt{0x10040008}$\\
        $\$t0$ & $1$
      \end{tabular}
      \hfill
      \begin{tabular}{l|c|}
        \cline{2-2}
        $\texttt{0x10040000}$ & 1\\
        & 2\\
        \cline{2-2}
        $\texttt{0x10040008}$ & 1\\
        & $\texttt{0x10040000}$\\
        \cline{2-2}
      \end{tabular}
      \hfill
    \end{table}

  \item $\$a0 = \texttt{0x10040000}$\\
    Tas:\\
    \begin{tabular}{l| c |}
      \cline{2-2}
      $\texttt{0x10040000}$ & 1 \\
      & $\texttt{0x10040008}$ \\ 
      \cline{2-2}
      $\texttt{0x10040008}$ & 1 \\
      & $\texttt{0x10040010}$ \\
      \cline{2-2}
      $\texttt{0x10040010}$ & 1\\
      & $3$\\
      \cline{2-2}
    \end{tabular}

  \item récupère la valeur $n$ de $\texttt{Somme (Somme n)}$  dans le registre $\$v0$
    \begin{lstlisting}[texcl=false, mathescape=false]
    lw $v0, 4($a0)
    lw $v0, 4($v0)
  \end{lstlisting}

  \item code de la fonction f:
    \begin{lstlisting}[texcl=false, mathescape=false]
      f:
        beqz  $a0, N
        lw    $v0, 4($a0)
        addi  $v0, $v0, 1
        jr    $ra
        N:
          li $v0, -1
          jr $ra
    \end{lstlisting}

  \newpage
  \item Code:
    \begin{lstlisting}[texcl=fasle, mathescape=false]
      bnez    $a0, some
      
      bnez    $a1, incompatible
      li      $v0, 1
      b       fin
    some:
      beqz    $a1, incompatible

      lw      $a0, 4($a0)
      lw      $a1, 4($a0)
      seq     $v0, $a0, $a1
      b fin
    incompatible:
      li      $v0, 0
    fin:
    \end{lstlisting}
\end{enumerate}
\end{exo}
\end{document}


