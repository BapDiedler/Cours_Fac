\documentclass{article}


\usepackage[utf8]{inputenc}
\usepackage[T1]{fontenc}
\usepackage[french]{babel}
\usepackage{listings}
\usepackage{xcolor}
\usepackage{ntheorem}
\usepackage{amsmath}
\usepackage{amssymb}
\usepackage[ a4paper, hmargin={1cm, 1cm}, vmargin={2cm, 2cm}]{geometry}


\usepackage{tikz}
\usetikzlibrary{automata, arrows.meta, positioning}

\theoremstyle{plain}
\theorembodyfont{\normalfont}
\theoremseparator{~--}
\newtheorem{exo}{Exercice}%[section]


\begin{document}
\begin{center}
    \large\sc Feuille d'exercice 3
    \end{center}

\begin{exo} 2 * (2 * 3 + 1) * 3:

    \begin{tikzpicture}
        \node [align=center] (1) at (0, 0) {2};
        \node [align=center] (2) at (0.5, 0) {*};
        \node [align=center] (3) at (1, 0) {(};
        \node [align=center] (4) at (1.5, 0) {2};
        \node [align=center] (5) at (2, 0) {*};
        \node [align=center] (6) at (2.5, 0) {3};
        \node [align=center] (7) at (3.5, 0) {+};
        \node [align=center] (8) at (4, 0) {1};
        \node [align=center] (9) at (5.5, 0) {)};
        \node [align=center] (10) at (6, 0) {*};
        \node [align=center] (11) at (6.5, 0) {3};

        \node [align=center] (12) at (4, 1) {A};
        \node [align=center] (13) at (4.5, 1) {M'};
        \node [align=center] (14) at (4.5, 0.25) {$\varepsilon$};
        \draw (13) -- (14);
        \draw (8) --  (12);
        \draw (12) -- (4, 1.5) -- (4.5, 1.5) -- (13);
        \node [align=center] (15) at (4.25, 2) {M};
        \draw (4.25, 1.5) -- (15);

        \node [align=center] (16) at (5, 2) {E'};
        \node [align=center] (17) at (5, 1) {$\varepsilon$};
        \draw (16) -- (17);
        \draw (7) -- (3.5, 2.5) -- (4.25, 2.5) -- (15);
        \draw (4.25, 2.5) -- (5, 2.5) -- (16);
        \node [align=center] (18) at (4.5, 3) {E'};
        \draw (4.5, 2.5) -- (18);

        \node [align=center] (12) at (2.5, 1) {A};
        \node [align=center] (13) at (3, 0.25) {$\varepsilon$};
        \node [align=center] (14) at (3, 1) {M'};

        \draw (6) -- (12);
        \draw (13) -- (14);
        \draw (5) -- (2, 1.5) -- (2.5, 1.5) -- (12);
        \draw (2.5, 1.5) -- (3, 1.5) -- (14);
        \node [align=center] (15) at (2.5, 2) {M'};
        \draw (2.5, 1.5) -- (15);
        \node [align=center] (16) at (1.5, 1.5) {A};
        \draw (4) -- (16);
        \draw (16) -- (1.5, 2.5) -- (2.5, 2.5) -- (15);
        \node [align=center] (17) at (2, 3) {M};
        \draw (2, 2.5) -- (17);

        \draw (17) -- (2, 3.5) -- (4.5, 3.5) -- (18);
        \node [align=center] (18) at (3.25, 4) {E}; 
        \draw (3.25, 3.5) -- (18);
        \draw (3) -- (1, 4.5) -- (3.25, 4.5) -- (18);
        \draw (3.25, 4.5) -- (5.5, 4.5) -- (9);
        \node (0) at (3.25, 5) {A};
        \draw (3.25, 4.5) -- (0);

        \node [align=center] (12) at (7, 4) {M'};
        \node [align=center] (13) at (7, 3) {$\varepsilon$};
        \draw (12) -- (13);
        \node (14) at (6.5, 4) {A};
        \draw (11) -- (14) -- (6.5, 4.5) -- (7, 4.5) -- (12);
        \draw (10) -- (6, 4.5) -- (6.5, 4.5);
        \node [align=center] (14) at (6.5, 5) {M'};
        \draw (6.5, 4.5) -- (14);

        \draw (0) -- (3.25, 5.5) -- (6.5, 5.5) -- (14);

        \draw (2) -- (0.5, 5.5) -- (3.25, 5.5);
        \node (15) at (3.5, 6) {M'};
        \draw (15) -- (3.5, 5.5);

        \node (16) at (0, 6) {A};
        \draw (1) -- (16) -- (0, 6.5) -- (3.5, 6.5) -- (15);
        \node [align=center] (17) at (1.75, 7) {M};
        \draw (1.75, 6.5) -- (17);
        \node [align=center] (18) at (7.5, 7) {E'};
        \node [align=center] (19) at (7.5, 6.25) {$\varepsilon$};
        \draw (19) -- (18) -- (7.5, 7.5) -- (1.75, 7.5) -- (17);
        \node (20) at (4.625, 8) {E};
        \draw (20) -- (4.625, 7.5);
    \end{tikzpicture}
\end{exo}

\begin{exo} Gramaires et associativité
\begin{enumerate}
    \item Les deux gramaires qui décrivent des langages différents sont $G_2$ et $G_3$
        \begin{itemize}
            \item $G_2$ force l'utilisation de parenthèse à enchaînement d'adition ou de multiplication.
            \item $G_4$ interdit toutes les parenthèse supreflues.
        \end{itemize}
    \item  flemme
    \item On peut ajouter $E \rightarrow E - M$ dans $G_0$ ou $G_1$
\end{enumerate}
\end{exo}

\begin{exo} Analyse ascendante
\begin{enumerate}
  \item $(1 + 2) + (3 + 4)$\\
    \begin{tabular}{| l | r | l |}
      \hline
      Pile & Entée & Action \\
      \hline
      $\varepsilon$ & $(1 + 2) + (3 + 4)$ & S \\ 
      $($ & $1 + 2) + (3 + 4)$ & S\\
      $(1$ & $ + 2) + (3 + 4)$ & $R[E \to n]$\\
      $(E$ & $ + 2) + (3 + 4)$ & S\\
      $(E + $ & $ 2) + (3 + 4)$ & S\\
      $(E + 2$ & $) + (3 + 4)$ & $R[E \to n]$\\
      $(E + E$ & $) + (3 + 4)$ & $R[E \to E + E]$\\
      $(E$ & $) + (3 + 4)$ & S\\
      $(E)$ & $ + (3 + 4)$ & $R[E \to (E)$\\
      $E$ & $+ (3 + 4)$ & S\\
      $E + $ & $(3 + 4)$ & S\\
      $E + ($ & $3 + 4)$ & S\\
      $E + (3$ & $ + 4)$ & $R[E \to n]$\\
      $E + (E$ & $ + 4)$ & S\\
      $E + (E +$ & $ 4)$ & S\\
      $E + (E + 4$ & $)$ & $R[E \to n$\\
      $E + (E + E$ & $)$ & $R[E \to E + E]$\\
      $E + (E$ & $)$ & S\\
      $E + (E)$ & $\emptyset$ & $R[E \to (E)]$\\
      $E + E$ & $\emptyset$ & $R[E \to E + E]$\\
      $E$ & $\emptyset$ & succès\\
      \hline
    \end{tabular}
\newpage
  \item (1 + 2)(3)\\
    \begin{tabular}{| l | r | l |}
      \hline
      Pile & Entée & Action \\
      \hline
      $\varepsilon$ & $(1 + 2) (3)$ & S \\ 
      $($ & $1 + 2) (3)$ & S\\
      $(1$ & $ + 2) (3)$ & $R[E \to n]$\\
      $(E$ & $ + 2) (3)$ & S\\
      $(E + $ & $ 2) (3)$ & S\\
      $(E + 2$ & $) (3)$ & $R[E \to n]$\\
      $(E + E$ & $) (3)$ & $R[E \to E + E]$\\
      $(E$ & $) (3)$ & S\\
      $(E)$ & $(3)$ & $R[E \to (E)]$\\
      $E$ & $(3)$ & échec\\
    \hline
    \end{tabular}
\end{enumerate}
\end{exo}
    

\begin{exo} Analyse descendante
\begin{enumerate}
  \item
    Annulables: 

    \begin{tabular}{c | c | c | c |}
      & S & E & L\\
      \hline
      0. & F & F & F \\
      1. & F & F & V \\
      2. & F & F & V \\
\end{tabular}

    \begin{align*}
      Permier(S) &= Permier(E)\\
      Premier(E) &= \left\{sym, (\right\}\\
      Permier(L) &= Permier(E)
    \end{align*}

    \begin{tabular}{c|c|c|c|}
      & S & E & L \\
      \hline
      0. & $\emptyset$ & $\emptyset$ & $\emptyset$ \\
      1. & $\emptyset$ & sym, (& $\emptyset$ \\
      2. & sym, ( & sym, ( & sym, ( \\
      3. & sym, ( & sym, ( & sym, ( \\
      \hline
    \end{tabular}

    \begin{align*}
      Suivants(S) &= \emptyset \\
      Suivants(E) &= \left\{\#\right\} \cup Premier(L) 
                          \cup Suivants(L)\\
      Suivants(E) &= \left\{)\right\} \cup Suivants(L)\\
    \end{align*}

  \begin{tabular}{c|c|c|c|}
      & S & E & L \\
      \hline
      0. & $\emptyset$ & $\emptyset$ & $\emptyset$ \\
      1. & $\emptyset$ & \#, sym, ( & ) \\
      2. & $\emptyset$ & \#, sum, (, ) & ) \\
      3. & $\emptyset$ & \#, sym, (, ) & ) \\
      \hline
    \end{tabular}

  \item Table LL(1):

    \begin{tabular}{c|c|c|c|c|}
      & sym & ( & ) & \# \\
      \hline
      S & E \# & E \# & & \\
      \hline
      E & sym & ( L ) & & \\
      \hline
      L & E L & EL & $\varepsilon$ & \\
      \hline
    \end{tabular}

\end{enumerate}
\end{exo}
\end{document}
