\documentclass{article}

\usepackage[utf8]{inputenc}
\usepackage[T1]{fontenc}
\usepackage[french]{babel}
\usepackage{ntheorem}
\usepackage{amsmath}
\usepackage{amssymb}
\usepackage{listings}
\usepackage[ a4paper, hmargin={2cm, 2cm}, vmargin={3cm, 3cm}]{geometry}

\usepackage{tikz}

\theoremstyle{plain}
\theorembodyfont{\normalfont}
\theoremseparator{~--}
\newtheorem{exo}{Exercice}%[section]

\pagestyle{empty}
\lstset{language=caml}


\begin{document}
\begin{center}
    \large\sc Feuille d'exercice 5
\end{center}

\begin{exo} Forme clausale
\begin{enumerate}
    \item \begin{align*}
        F_1 &=\forall x, (p(x) \Rightarrow \exists y, \forall x, q(x, y)) \\
            &\equiv \forall x, (\neg p(x) \vee (\exists y, \forall x, q(x, y))) \\
            &\leadsto  \forall x', (\neg p(x') \vee (\forall x, q(x, a))) \\
            &\equiv \forall x', x (\neg p(x') \vee (q(x, a))) \\
    \end{align*}
    Forme clausale: $\left\{\neg p(x') \vee q(x, a)\right\}$

    \item \begin{align*}
        F_2 &= ((\exists x, (p(x) \Rightarrow r(x))) \vee \forall y, p(y)) \wedge 
                    \forall x, \exists y, (r(y) \Rightarrow p(x)) \\
            &\equiv (\exists x, (\neg p(x) \vee r(x)) \vee \forall y, p(y)) \wedge 
                    \forall x, \exists y, (\neg r(y) \vee p(x)) \\
            &\leadsto (\neg p(a) \vee r(a) \vee \forall y, p(y)) \wedge 
                    \forall x, (\neg r(f(x)) \vee p(x)) \\
            &\equiv \forall y, x, (\neg p(a) \vee r(a) \vee p(y)) \wedge 
                    (\neg r(f(x)) \vee p(x)) \\
    \end{align*}
    Forme clausale: $\left\{\neg p(a) \vee r(a) \vee p(y), \neg r(f(x)) \vee p(x)\right\}$

    \item \begin{align*}
        F_3 &= ((\forall x, (p(x) \Rightarrow \exists y, q(y))) \Rightarrow \exists z, r(z)) 
                \Rightarrow \exists u, s(u) \\
            &\equiv \neg ((\forall x, (p(x) \Rightarrow \exists y, q(y))) \Rightarrow 
                \exists z, r(z)) \vee \exists u, s(u) \\
            &\equiv ((\forall x, (p(x) \Rightarrow \exists y, q(y))) \wedge 
                \forall z, \neg r(z)) \vee \exists u, s(u) \\
            &\equiv ((\forall x, (\neg p(x) \vee \exists y, q(y))) \wedge
                \forall z, \neg r(z)) \vee \exists u, s(u) \\
            &\leadsto ((\forall x, (\neg p(x) \vee q(a))) \wedge
                \forall z, \neg r(z)) \vee s(b) \\
            &\equiv \forall x, z, ((\neg p(x) \vee q(a)) \wedge
                \neg r(z)) \vee s(b) \\
            &\equiv \forall x, z, (\neg p(x) \vee q(a) \vee s(b)) \wedge
                (\neg r(z) \vee s(b)) \\
    \end{align*}
    Forme clausale: $\left\{\neg p(x) \vee q(a) \vee s(b), \neg r(z) \vee s(b) \right\}$
\end{enumerate}
\end{exo}

\begin{exo} Satisfiabilité
\begin{enumerate}
    \item $B=(\forall x, P(x)) \vee (\exists x, \neg P(x) \wedge \neg Q(x)) 
            \vee (\forall x, Q(x))$
    \item \begin{itemize}
        \item Oui
        \item Oui
        \item Oui
    \end{itemize}
    \item \begin{align*}
        C   &=(\forall x, P(x)) \vee (\neg P(a) \wedge \neg Q(a)) 
                \vee (\forall x, Q(x)) \\
            &\equiv \forall x, y, P(x) \vee (\neg P(a) \wedge \neg Q(a)) 
                \vee Q(y)
    \end{align*}
    \item  \begin{itemize}
        \item Non
        \item Oui
        \item Oui
    \end{itemize}
    \item \begin{itemize}
        \item C est toujours Vraie dans un modèle dont le domaine a exactement un élément
        \item Pareil pour A car on a $C \models B  \models A$
    \end{itemize}
    \item  Oui
    \item $\neg A \equiv (\exists x, \neg P(x)) \wedge (\forall x, P(x) \vee Q(x)) 
        \wedge (\exists x, \neg Q(x))$.

        Après skolémisation on à les trois closes:\\
        $C_1 = \neg P(a)$, $C_2 = P(x) \vee Q(x)$, $C_3 = \neg Q(b)$

    \item $\mathcal{D}_H = \left\{a, b\right\} $\\
        $\mathcal{B}_H = \left\{P(a), P(b), Q(a), Q(b)\right\} $

    \item $\neg A$ est satsifiable avec $\left\{P(b) \leftarrow V, Q(a), \leftarrow V\right\}$\\
        Comme $\neg A$ est satsifiable $A$ n'est pas valide.
\end{enumerate}
\end{exo}

\begin{exo} Forme de Herbrand
    
\end{exo}

\begin{exo} Connecteur propositionnel IF
\begin{enumerate}
    \item \begin{description}
        \item[$\neg P$]: $\textbf{IF}(P, \bot, \top)$s
        \item[$P \wedge Q$]: $\textbf{IF}(P, Q, \bot)$
        \item[$P \vee Q$]: $\textbf{IF}(P, \top, Q)$
        \item[$P \Rightarrow Q$]: $\textbf{IF}(P, Q, \top)$
    \end{description}

    \item \begin{align*}
        valIF(I, F) = \begin{cases}
            valIF(I, \top) &= V \\
            valIF(I, \bot) &= F \\
            valIF(I, p) &= I(p) \\
            valIF(I, \textbf{IF}(P, Q, R)) &= \text{si } valIF(I, P) \\
                                        &\text{ alors } valIF(I, Q) \\
                                        &\text{ sinon } valIF(I, R)
        \end{cases}
    \end{align*}

    \item \begin{itemize}
        \item Si $I \models P$: $\textbf{IF}(P, Q, R) \equiv Q$
            \\ $A \equiv \textbf{IF}(Q, S, T)$ et $B \equiv \textbf{IF}(Q, S, T)$
        \item  SI $I \not\models P$: $\textbf{IF}(P, Q, R) \equiv R$
            \\ $A \equiv \textbf{IF}(R, S, T)$ et $B \equiv \textbf{IF}(R, S, T)$
    \end{itemize}

    \item \begin{align*}
        IFn(P, Q, R) = \begin{cases}
            IFn(\top, Q, R) &= Q \\
            IFn(\bot, Q, R) &= R \\
            IFn(p, Q, R) &= \textbf{IF}(p, Q, R) \\
            IFn(\textbf{IF}(A, B, C), Q, R) &= IFn(A, 
                \textbf{IF}(B, Q, R), \textbf{IF}(C, Q, R))
        \end{cases}
    \end{align*}
    \begin{align*}
        norm(P) = \begin{cases}
            norm(\top) &= \top \\
            norm(\bot) &= \bot \\
            norm(p) &= p \\
            norm(\textbf{IF}(P, Q, R)) &= Ifn(P, norm(Q), norm(R))
        \end{cases}
    \end{align*}

    \item \begin{enumerate}
        \item $\textbf{IF}(\textbf{IF}(\textbf{IF}(p, q, \top), p, \top), p, \top)$
        \item \begin{align*}
            \textbf{IF}(\textbf{IF}(\textbf{IF}(p, q, \top), p, \top), p, \top) 
                &\equiv \textbf{IF}(\textbf{IF}(p, \textbf{IF}(q, p, \top), p, \top), p, \top) \\
                &\equiv \textbf{IF}(p, \textbf{IF}(
                        \textbf{IF}(q, p, \top), p, \top), \textbf{IF}(p, p, \top)) \\
                &\equiv \textbf{IF}(p, \textbf{IF}(
                    q, \textbf{IF}(p, p, \top), p), \textbf{IF}(p, p, \top)) \\
        \end{align*}

        \item Arbre:\\
        \begin{tikzpicture}[level distance=1cm, level 1/.style={sibling distance=2cm},
                    level 2/.style={sibling distance=1cm},
                    level 3/.style={sibling distance=1cm}]
            \node (root) {IF(p)}
                child { node {IF(q)}
                        child { node {IF(p)}
                                child { node {p} }
                                child { node {V} }
                            }
                        child { node {p} }
                    }
                child { node {IF(p)}
                        child { node {p} }
                        child { node {V} }
                    }
            ;
        \end{tikzpicture}

        \item Pour montrer que A est valide on regarde les branches qui ne 
            débouche pas sur une feuille contenant $\top$ (sinon cas trivial):\\
            Feuilles de gauche à droite:
            \begin{itemize}
                \item Première feuille: pour y aller il faut avoir $p$ vrai donc peut importe q
                    la formule serra vrai
                \item Deuxième feuille: Pareil que pour la deuxième pour y accèder il faut 
                    avoir $p$ vrai
                \item Troisième feuille: Jamais accéssible car pour alleer dans 
                le premier noeuds il faut avoir $\neg p$ et après il faut avoir $p$
            \end{itemize}

            Donc peut importe notre interpretation la formule serra toujours valide.

    \end{enumerate}

    \item On normalise puis on simplifie l'arbre (varaible apparaissent 
        qu'une seul foit par branche) et après il faut vérfier que les feuilles soient toutes vrai
\end{enumerate}
\end{exo}

\begin{exo}Arbre de décision bianire
\begin{enumerate}
    \item \begin{align*}
        vald(I, t) = \begin{cases}
            vald(I, Bool(b)) &= b\\
            vald(I, \textbf{IF}(x, P, Q)) &=  \text{si } I(x) \\ 
                        &\text{ alors } vald(I, P) \\
                        &\text{ sinon } vald(I, Q)
        \end{cases}
    \end{align*}
    \item \begin{align*}
        forme(t) = \begin{cases}
            forme(Bool(true)) &= \top \\
            forme(Bool(false)) &= \bot \\
            forme(\textbf{IF}(p, A, B)) &= (p \wedge forme(A)) \vee (\neg p \wedge forme(B))
        \end{cases}
    \end{align*}

    \item \begin{align*}
        notd(t) = \begin{cases}
            notd(Bool(true)) &= Bool(false) \\
            notd(Bool(false)) &= Bool(true) \\
            notd(\textbf{IF}(p, A, B)) &= \textbf{IF}(p, notd(A), notd(B))
        \end{cases}
    \end{align*}

    \item \begin{align*}
        opd(t, u) = \begin{cases}
            opd(Bool(b_1), Bool(b_2)) &= Bool(b_1 \text{ op } b_2) \\
            opd(IF(x, A, B), Bool(b)) &= \textbf{IF}(x, opd(A, Bool(b)), opd(B, Bool(b))) \\
            opd(Bool(b), IF(x, A, B)) &= \textbf{IF}(x, opd(A, Bool(b)), opd(B, Bool(b))) \\
            opd(\textbf{IF}(x, A, B), \textbf{IF}(y, C, D)) &= \\
                            &\text{if } x = y \text{ alors} \\
                                &\hspace{5mm} \textbf{IF}(x, opd(A, C), opd(B, D)) \\
                            &\text{sinon si } x < y \text{ alors} \\
                                &\hspace{5mm} \textbf{IF}(x, opd(A, IF(y, C, D)), opd(B, IF(y, C, D))) \\
                            &\text{sinon } \\
                                &\hspace{5mm} \textbf{IF}(y, opd(C, IF(y, A, B)), opd(D, IF(y, A, B)))
        \end{cases}
    \end{align*}

    \item \begin{itemize}
        \item valide: Toutes les feuilles $V$
        \item satsifiable: Au moin une feuille $V$
        \item insatsifaible: Toutes les feuilles $F$
    \end{itemize}

    \item Il faut trouver un chemin qui donne vers une feuille vraie
\end{enumerate}
\end{exo}

\begin{exo} Diagramme de décision bianire
\begin{enumerate}
    \item OBDD:\\
        \begin{tikzpicture}
            \node at (0, 1) {$(z \Leftrightarrow t)$};
            \node [draw, circle](z) at (0, 0) {z};
            \node [draw, circle](t1) at (-1, -1) {t};
            \node [draw, circle](t2) at (1, -1) {t};
            \node [draw](V) at (-1, -2) {V};
            \node [draw](F) at (1, -2) {F};

            \draw[] (z) -- (t1);
            \draw[] (t1) -- (V);
            \draw[] (t2) -- (F);
            \draw[dashed] (z) -- (t2);
            \draw[dashed] (t1) -- (F);
            \draw[dashed] (t2) -- (V);

            \node at (10, 1) {$(x \Leftrightarrow y) \wedge (z \Leftrightarrow t)$};
            \node [draw, circle](x) at (10, 0) {x};
            \node [draw, circle](y1) at (9, -1) {y};
            \node [draw, circle](y2) at (11, -1) {y};
            \node [draw, circle](z) at (8, -2) {z};
            \node [draw, circle](t1) at (7, -3) {t};
            \node [draw, circle](t2) at (9, -3) {t};
            \node [draw](V) at (8, -5) {V};
            \node [draw](F) at (10, -5) {F};

            \draw[] (x) edge (y1) (y1) edge (z) (z) edge (t1) (t1) edge (V);
            \draw[dashed] (x) edge (y2) (y2) edge (z) (z) edge (t2) (t2) edge (V);
            \draw[dashed] (y1) -- (F);
            \draw[] (y2) -- (F);
            \draw[] (y1) -- (z);
            \draw[] (x) -- (y1);
            \draw[] (t2) -- (F);
            \draw[dashed] (t1) -- (F);
        \end{tikzpicture}

    \item OBDD annoté:\\
        \begin{tikzpicture}[minimum size=0.5cm]
            \node at (0, 1) {$(z \Leftrightarrow t)$};
            \node [draw](z) at (0, 0) {$(z \wedge t) \vee (\neg z \wedge \neg t)$};
            \node [draw](t1) at (-1, -1) {$t$};
            \node [draw](t2) at (1, -1) {$\neg t$};
            \node [draw](V) at (-1, -2) {V};
            \node [draw](F) at (1, -2) {F};

            \draw[] (z) -- (t1);
            \draw[] (t1) -- (V);
            \draw[] (t2) -- (F);
            \draw[dashed] (z) -- (t2);
            \draw[dashed] (t1) -- (F);
            \draw[dashed] (t2) -- (V);

            \node at (10, 1) {$(x \Leftrightarrow y) \wedge (z \Leftrightarrow t)$};
            \node [draw](x) at (10, 0) {$(a \wedge y \wedge x) \vee (a \wedge \neg y \wedge \neg x)$};
            \node [draw](y1) at (9, -1) {$a \wedge y$};
            \node [draw](y2) at (11, -1) {$a \wedge \neg y$};
            \node [draw](z) at (7, -2) {$a=(z \wedge t) \vee (\neg z \wedge \neg t)$};
            \node [draw](t1) at (7, -3) {$t$};
            \node [draw](t2) at (9, -3) {$\neg t$};
            \node [draw](V) at (8, -5) {V};
            \node [draw](F) at (10, -5) {F};

            \draw[] (x) edge (y1) (y1) edge (z) (z) edge (t1) (t1) edge (V);
            \draw[dashed] (x) edge (y2) (y2) edge (z.north east) (z) edge (t2) (t2) edge (V);
            \draw[dashed] (y1.349) -- (F);
            \draw[] (y2) -- (F);
            \draw[] (y1) -- (z);
            \draw[] (x) -- (y1);
            \draw[] (t2) -- (F);
            \draw[dashed] (t1) -- (F);
        \end{tikzpicture}

    \item On va résonner par double implication:
        \begin{description}
            \item[$\Leftarrow$]: Si deux formules sont représenté par le même OBDD alors elles
                ont la même table de vérité donc elles sont équivalentes.
            
            \item[$\Rightarrow$]: Supposons qu'on ait deux formules A et B équivalentes.
                
        \end{description}
\end{enumerate}
\end{exo}
\end{document}
