\documentclass{article}

\usepackage[utf8]{inputenc}
\usepackage[T1]{fontenc}
\usepackage[french]{babel}
\usepackage{ntheorem}
\usepackage{amsmath}
\usepackage{amssymb}
\usepackage{listings}
\usepackage[ a4paper, hmargin={2cm, 2cm}, vmargin={3cm, 3cm}]{geometry}

\usepackage{tikz}

\theoremstyle{plain}
\theorembodyfont{\normalfont}
\theoremseparator{~--}
\newtheorem{exo}{Exercice}%[section]

\pagestyle{empty}
\lstset{language=caml}


\begin{document}
\begin{center}
    \large\sc Feuille d'exercice 4
\end{center}

\begin{exo}
\begin{enumerate}
    \item $\mathfrak{D} = \left\{a,b\right\}$\\
        Base de Herbrand: $P(a), P(b), Q(a), Q(b)$ $2^4$ interprétation\\
        $\left\{P(a) \wedge Q(a) \wedge (\neg P(a) \vee Q(b)), 
                P(b) \wedge Q(b) \wedge (\neg P(a) \vee Q(b))\right\}$ 
                
        Or ces deux formules ne peuvent pas être vrai en même temps donc la formule est insatisfiable
    \item $\mathfrak{D} = \left\{a,b\right\}$ \\
        Base de Herbrand: $P(a), P(b), Q(a), Q(b)$\\
        $\left\{(P(a) \vee Q(a)) \wedge \neg P(a) \wedge \neg Q(b), 
                (P(b) \vee Q(b)) \wedge \neg P(a) \wedge \neg Q(b)\right\}$

        Satisfiable si $P(a) = F, P(b) = V, Q(a) = V, Q(b) = F$

    \item Signature: $f$ arité 1, on ajoute une constant $a$\\
        $\mathfrak{D} = \left\{a, f(a), f(f(a)), \ldots, f(\ldots f(a))\right\}$ \\
        Base de Herbrand: $P(a), P(f(a)), \ldots, P(f^n(a))$\\
        $\left\{P(a), \wedge \neg P(f(a)), P(f(a)) \wedge \neg P(f^2(a)), \ldots\right\}$

        insatisfiable car si $P(a) \wedge \neg P(f(a))$ 
        est vrai alors $P(f(a)) \wedge \neg P(f^2(a))$ est faux car $P(f(a)), \neg P(f(a))$ 
        doivent être vrai.

    \item Signature: $s$ arrité 1 avec les constante a, b etc\\
        $\mathfrak{D} = \left\{a, s(a), \ldots, s^n(a)\right\}$\\
        Base de Herbrand: $B = \left\{R(s^k(a), s^l(a) | k, l \in \mathbb{N})\right\}$

        satisfiable: $R_H = \left\{R(S^k(a), s^l(a)) | l > k\right\} $
        
\end{enumerate}
\end{exo}

\begin{exo} Base de Herbrand sur ensembles de formules
\begin{enumerate}
  \item $\left\{\forall x, (P(x) \vee Q(x) \vee R(x)); \neg P(a); \neg Q(b); \neg R(c)\right\}$\\
    Signature = $\left\{a, b, c\right\}$\\
    $\mathfrak{D} = \left\{a, b, c\right\}$\\
    Base de Herbrand: $P(a), P(b), P(c), Q(a), Q(b), Q(c), R(a), R(b), R(c)$\\
    $\left\{(P(a) \vee Q(a) \vee R(a)), (P(b) \vee Q(b) \vee R(b)), (P(c) \vee Q(c) \vee R(c)),
      \neg P(a), \neg Q(b), \neg R(c)\right\}$\\
    Satsifiable si $P(b) = V, Q(c) = V, R(a) = V$

  \item $\left\{\forall x, P(x); \forall x, \neg Q(x); \forall x, (\neg P(f(x)) \vee Q(f(x)))\right\}$\\
    Signature = $f$ d'arrité 1 et une constant a\\
    $\mathfrak{D} = \left\{a, f(a), \ldots, f(\ldots f(a))\right\}$\\
    $\left\{P(f^n(a)); \neg Q(f^n(a)); \neg P(f^{n+1}(a)) \vee Q(f^{n+1}(a)) | n \in \mathbb{N}\right\}$
    
\end{enumerate}
\end{exo}

\begin{exo} Ocaml
\begin{lstlisting}
    let rec clauseb = function                    let rec fncb = function
        | Var n -> true                              | Var n -> true
        | Bot -> true                                | Bot -> true
        | Top -> true                                | Top -> true
        | Neg f -> (match f with                     | Neg f -> clauseb (Neg f)
                    |Var n -> True                   | Bin(f, Or, g) ->     
                    | _ -> false)                       clauseb f && clauseb g
        | Bin(f, Or, g) -> clauseb f && clause g     | Bin(f, Et, g) ->
        | Bin(f, Et, g) -> false                        fncb f && fncb g
        | Bin(f, Impl, g) -> false                   | Bin(f, Impl, g) -> false
\end{lstlisting}
\end{exo}

\begin{exo} Formes normales
\begin{enumerate}
    \item $p \wedge q$ forme normale conjonctive et disjonctives
    \item $p \vee \neg q$ clauses, forme normale conjonctive et disjonctives
    \item $\neg(p \vee q)$ rien
    \item $p \wedge q \vee r$ forme normale disjonctives
\end{enumerate}
\end{exo}

\begin{exo} Formes normales à partir d'une table de vérité
    \begin{description}
        \item[$P$]:\begin{description}
            \item[$FNC$]: $(\neg a \vee \neg b \vee c) \wedge
                (\neg a \vee b \vee \neg c) \wedge
                (a \vee \neg b \vee \neg c)$
            \item[$FND$]: $(a \wedge b \wedge c) \vee 
                (a \wedge \neg b \wedge c) \vee 
                (\neg a \wedge b \wedge c) \vee 
                (\neg a \wedge \neg b \wedge c) \vee 
                (\neg a \wedge \neg b \wedge \neg c)$

                $\equiv c \vee (\neg a \wedge \neg b \wedge \neg c)$

                $\equiv c \vee (\neg a \wedge \neg b)$
        \end{description} 
    \end{description}
\end{exo}

\begin{exo}Formes normales
\begin{enumerate}
    \item 
    \begin{enumerate}
        \item \begin{align*}
            (\neg p \wedge \neg q \wedge r) \Rightarrow (r \vee s)  
                    &\equiv \neg(\neg p \wedge \neg q \wedge r) \vee (r \vee s)\\
                    &\equiv (p \wedge q \wedge \neg r) \vee r \vee s          &\text{FND} \\
                    &\equiv (p \vee r \vee s) \wedge (q \vee r \vee s) \wedge (\neg r \vee r \vee s) \\
                    &\equiv (p \vee r \vee s) \wedge (q \vee r \vee s)          &\text{FNC}
        \end{align*}
        
        \item \begin{align*}
             p \Rightarrow ((\neg q \vee r) \Rightarrow s)
                &\equiv \neg p \vee (\neg(\neg q \vee r) \vee s) \\
                &\equiv \neg p \vee (q \wedge \neg r) \vee s     &\text{FNC} \\
                &\equiv (\neg p \vee q \vee s) \wedge (\neg p \vee \neg r \vee s)  &\text{FND}
        \end{align*}

        \item 
        \item $(p \Rightarrow q) \wedge \neg q \wedge \neg(q \Rightarrow p)$
    \end{enumerate}

    \item \begin{enumerate}
        \item Vrai: r
        \item Vrai: s
    \end{enumerate}

    \item \begin{itemize}
        \item Oui, si il n'y a que des et ou que de ou
        \item Non, on peut factoriser certaines clauses
        \item 
    \end{itemize}
\end{enumerate}
\end{exo}
\end{document}
